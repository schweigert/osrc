\chapter{Introdução}
\label{cap1}


\section{Fundamentação Teórica}


A fim de orientar o análise encontrada neste trabalho, se faz necessário a descrição e pesquisa por fundamentação teórica dos conceitos básicos relevantes no contexto de microsserviços e implantação dessas arquiteturas.
%
Por esse motivo, esta seção irá introduzir os conceitos de microsserviços, containers, docker e implantação utilizando a técnica de docker swarm.


\subsection{Microserviços: Definição e Funcionamento}

Entende-se por microsserviço, aplicações que executam operações menores de um macrosserviço, da melhor forma possível~\cite{stephenclarkewillson2017, Newman2015Feb}.
%
O objetivo de uma arquitetura de microsserviços é funcionar separadamente de forma autônoma, contendo baixo acoplamento~\cite{Newman2015Feb}.
%
Seu funcionamento deve ser desenhado para permitir alinhamentos de alta coesão e baixo acoplamento entre os demais microsserviços existentes em um macrosserviço~\cite{8169955}.



Arquiteturas de microsserviços iniciam uma nova linha de desenvolvimento de aplicações preparadas para executar sobre nuvens computacionais, promovendo maior flexibilidade, escalabilidade, gerenciamento e desempenho, sendo a principal escolha de arquitetura de grandes empresas como Amazon, Netflix e LinkedIn~\cite{7830692,7515686}.
%
Um microsserviço é definido pelas seguintes características~\cite{8169955}:



\begin{itemize}
  \item Deve possibilitar a implementação como uma peça individual do macrosserviço.
  \item Deve funcionar individualmente.
  \item Cada serviço deve ter uma interface. Essa interface deve ser o suficiente para utilizar o microsserviço.
  \item A interface deve estar disponível na rede para chamada de processamento remoto ou consulta de dados.
  \item O serviço pode ser utilizado por qualquer linguagem de programação e/ou plataforma.
  \item O serviço deve executar com as dependências mínimas.
  \item Ao agregar vários microsserviços, o macrosserviço resultante poderá prover funcionalidades complexas.
\end{itemize}



O microsserviço deverá ser uma entidade separada.
%
A entidade deve ser implantada como um sistema independente em um \ac{paas}.
%
Toda a comunicação entre os microsserviços de um macrosserviço será executada sobre a rede, a fim de reforçar a separação entre cada serviço.
%
As chamadas pela rede com o cliente ou entre os microsserviços será executada através de uma \ac{api}, permitindo a liberdade de tecnologia em que cada um será implementado~\cite{Newman2015Feb}.
%
Isso permite que o sistema contenha tecnologias distintas que melhor resolvam os problemas relacionados ao contexto deste microsserviço.
%
Isso pode ser visualizado na Figura~\ref{fig:microsservicos_tecnologias}.



\begin{figure}[htb!]
\caption{Microsserviços podem ter diferentes tecnologias}
\label{fig:microsservicos_tecnologias}
\includegraphics[height=4cm]{img/cap2/microsservicos_tecnologias.png}
\centering

Adaptado de:~\cite{Newman2015Feb}
\end{figure}



Uma arquitetura de microsserviços é escalável, como visível na Figura \ref{fig:microsservicos_escalabilidade}.
%
Ela permite o aumento do número de microsserviços sob demanda para suprir a necessidade de escalabilidade.
%
Este modelo computacional obtém maior desempenho, principalmente se executar sobre plataformas de computação elástica, na qual o orquestrador do macrosserviço pode aumentar o número de instâncias conforme a necessidade de requisições~\cite{Nadareishvili2016Aug}.



\begin{figure}[htb!]
\caption{Microsserviços são escaláveis}
\label{fig:microsservicos_escalabilidade}
\includegraphics[height=5cm]{img/cap2/microsservicos_escalabilidade.png}
\centering

Adaptado de:~\cite{Newman2015Feb}
\end{figure}



Microsserviços desenvolvidos para web utilizam arquitetura \ac{rest} baseado sobre o protocolo \ac{http}.
%
É uma boa prática utilizar o corpo com conteúdo da requisição e resposta no formato \ac{json} nas chamadas a uma \ac{api} de microsserviço web~\cite{Nadareishvili2016Aug}.

Entretanto, necessita-se de um método para garantir o seu isolamento, a qual, em sistemas Linux, podem ser garantidos utilizando Containers.
%
Por este motivo, se faz necessário descreve-lo.



\subsection{Containers}



Os container permitem ao desenvolvedor "empacotar" sua aplicação e todas as suas dependências(ex. bibliotecas), dessa forma, lhe é garantido que a aplicação terá o mesmo comportamento independentemente do hospedeiro Linux.

Um container Linux é um conjunto de processos que executam de forma isolada do restante do sistema. Esses processos utilizam arquivos providos de uma imagem, a qual lhe garante a compatibilidade a fim de evitar problemas de versionamento e conflitos de processos.
%
Essa separação pode ser visualizada na Figura~\ref{fig:container}.



\begin{figure}[htb!]
\caption{Containers sobre sistema linux}
\label{fig:container}
\includegraphics[height=4cm]{img/cap2/what-is-a-container.png}
\centering

Fonte:~\cite{container}
\end{figure}



Entretanto, se faz necessário uma ferramenta para gerenciamento de imagens e containers.
%
Entra em cena a partir de 2008 o \textit{Docker}, uma ferramenta que permite, de forma prática, a publicação de imagens no formato Dockerfile, além de contar com um gerenciador e repositório de imagens.


\subsection{Docker}

Docker é uma plataforma que nos permite "construir, emabarcar e rodar uma aplicação em qualquer lugar". Ele percorreu um longo caminho em um período de tempo incrivelmente curto e atualmente é considerado uma solução padrão para um dos apectos mais custosos do software: a implantação
~\cite{miellJan16}

Docker é uma ferramenta desenvolvida para facilitar o processo de criação, implantação e execução de aplicações por meio do uso de containers.
%
Pode ser pensado como um tipo de maquina virtual, porém diferentemente desta que necessita a criação de todo um sistema operacional virtual, o Docker, permite que as aplicações compartilhem o mesmo kernel Linux que o hospedeiro, reduzindo assim o tamanho da aplicação e obtendo um ganho de desempenho.


A tecnologia Docker usa o \textit{kernel} Linux, abstraindo sistemas como \textit{Cgroups} e \textit{namespaces} para segregar processos a fim que eles possam ser executados de forma independente.
%
O objetivo dos containers criados pelo Docker continua da mesma forma que  os containers Linux, conhecida como \textit{LXC}.
%
A diferença entre Docker e \textit{LXC} é relevante a escalabilidade, visto que containers Docker permitem multiplos processos executando juntamente a uma biblioteca, já o padrão \textit{LXC} permite somente um processo junto a sua biblioteca, consumindo mais recursos da máquina.
%
Essa comparação pode ser visualizada na Figura~\ref{fig:container_docker}.

\begin{figure}[htb!]
\caption{Tecnologia Docker em comparação aos containers Linux (LXC).}
\label{fig:container_docker}
\includegraphics[height=4cm]{img/cap2/traditional-linux-containers-vs-docker_0.png}
\centering

Fonte:~\cite{container}
\end{figure}



\subsection{Docker Swarm}

É uma ferramenta nativa do Docker que permite criar clusters de containers, chamado swarms, o que possibilitam a escalabilidade de recursos de acordo com a demanda(carga)~\cite{turnbullMarc17}
O que possibilita que diversos hospedeiros de Docker estejam inseridos no mesmo pool de recursos, facilitando assim a implantação de containers, uma vez que o Swarm disponibiliza uma API de integração que abstrai grande parte das
atividades necessárias a administração dos conteiners e promove um tipo de tolerancia a falhas


\begin{figure}[htb!]
\caption{Rede de Docker Swarm.}
\label{fig:docker_swarm}
\includegraphics[height=4cm]{img/cap2/docker_swarm.png}
\centering

Fonte:~\cite{container}
\end{figure}


O seu principal objetivo é resolver problemas de gerência de microsserviços, a qual antes eram resolvidos somente com \textit{Kubernets}, uma ferramenta criada pela Google em 2015.
%
Com Docker Swarm, pode-se ter um nó líder que gerenciará a rede, e nós trabalhadores.
%
Um exemplo de rede Docker Swarm pode ser visualizado na Figura~\ref{fig:docker_swarm}.

%\section{Histórico}

\subsection{Arquitetura Monolítica}

Arquitetura Monolítica, é uma arquitetura de desenvolvimento, na qual típicamente , apesar da complexidade dos sistemas ser quebrada ao se utilizar módulos, esses são projetados para a criação de um único executável, o qual possui todos os seus módulos executados em um mesma máquina.
Com o passar do tempo, o sistema cresce e tende a tornar-se cada vez mais complexo, o que gera diversos problemas em sua manutenção, por exemplo temos, a escalabilidade do sistema, que exige que o mesmo seja replicado inteiramente, mesmo que apenas uma parte desse seja necessária na nova instância, aumentando assim os custos.

Com a necessidade de escalabilidade, as arquiteturas de microsserviços obteram sucesso em grandes projetos comerciais como LinkedIn, Google e Youtube.

\begin{figure}[htb!]
\caption{Arquitetura Monolítica X Arquitetura de Microsserviços}
\label{fig:}
\includegraphics[height=4cm]{img/cap2/monoXmicro.png}
\centering
\end{figure}


%\section{Funcionamento}


\section{Boas práticas}
\begin{itemize}
	\item Não armazenar dados em containers, uma vez que esses podem ser parados, destruidos ou mesmo substituidos, se ncessário armazenar dados deve ser feito em um volume, com o cuidado de evitar que dois ou mais containers escrevam dados em um mesmo volume, o que poderia causar o corrompimento dos dados.
	\item Não criar imagens grandes, já que essas possuirão uma distribuição complexa, uma imagens de possuir apenas as bibliotecas e arquivos necessários para a execução da aplicação ou processo.
	\item Não executar mais de um processo/aplicação em um único container, pois tal comportamento acarretará em aumento da complexidade do gerenciamento e no número de logs.
	\item Não dos endereços IP dos containers, o endereço IP do container pode se alterado quando o mesmo é iniciado e parado. Caso seja necessária a comunicação entre a aplicação ou microsserviço e um outro container, é recomendado o uso de variáveis de ambiente para transmitir o hostname e porta corretos de um container para outro.
\end{itemize}
\cite{boasPraticas}



\section{Principais aplicações}

\subsection{Aplicações Web}

Microsserviços desenvolvidos para web utilizam arquitetura \ac{rest} baseado sobre o protocolo \ac{http}.
É uma boa prática utilizar o corpo com conteúdo da requisição e resposta no formato \ac{json} nas chamadas a uma \ac{api} de microsserviço web~\cite{Nadareishvili2016Aug}.

\subsection{Streaming}
Devido as suas propriedades os microsserviços, possibilitam que operações de streaming(cloud-based) possuam maior elasticidade e escalabilidade do que uma implementação padrão(monolítica), além de possibilitar a redução de custos de operação.\cite{microservicosStreaming}
\subsection{Jogos}

Alguns exemplos de arquitetura de microsserviços para jogos \ac{mmorpg} são as arquiteturas apresentadas por Rudy (Figura~\ref{fig:rudy}), Salz (Figura~\ref{fig:salz}) e a arquitetura escrita por Knowles (Figura~\ref{fig:knowles}).


\begin{figure}[htb!]
  \caption{Arquitetura Rudy, utilizada no jogo Tibia.}
  \label{fig:rudy}
  \includegraphics[height=3.5cm]{arquiteturas/rudy.png}
  \centering

  Adaptado de:~\cite{matthiasrudy2011}
\end{figure}


\begin{figure}[htb!]
  \caption{Arquitetura Salz, utilizada no jogo Albion.}
  \label{fig:salz}
  \includegraphics[height=3.5cm]{arquiteturas/salz.png}
  \centering

  Adaptado de:~\cite{albion_online_unite}
\end{figure}%

\begin{figure}[htb!]
  \caption{Arquitetura Knowles, utilizada no jogo Guild Wars 2.}
  \label{fig:knowles}
  \includegraphics[height=5.5cm]{arquiteturas/knowles.png}
  \centering

  Adaptado de:~\cite{stephenclarkewillson2017}
\end{figure}

A arquitetura Rudy (Figura~\ref{fig:rudy}) é formada por um sistema cliente-servidor monolítico, na qual cada microsserviço individual gerencia um mundo mútuo dos demais gerenciadores de mundo~\cite{matthiasrudy2011}.
%
Essa arquitetura dificulta a escalabilidade, modificações e manutenção~\cite{8169955}, além de segregar a comunidade de jogadores em servidores menores~\cite{matthiasrudy2011}.
%
Inicialmente essa arquitetura foi pensada para ser um sistema Cliente-Servidor monolítico.
%
A arquitetura Rudy é uma arquitetura de microsserviços adaptada de um serviço cliente-servidor~\cite{matthiasrudy2011}.
%
O jogo Tibia\footnote[1]{Tibia: \url{http://www.tibia.com}}, operante sobre essa arquitetura, possui 68 mundos oficiais (Sendo 2 servidores de teste)~\cite{matthiasrudy2011}, com capacidade para 1.050 clientes em cada servidor, na qual encontra-se restringido pelo gerenciador de mundo.



A arquitetura Salz (Figura~\ref{fig:salz}) é formada por diversos microsserviços~\cite{albion_online_unite}.
%
O principal objetivo dessa arquitetura é modularizar o serviço visando melhorar a escalabilidade.
%
Ela é atualmente utilizada no jogo Albion Online\footnote[2]{Albion Online: \url{https://albiononline.com}}.
%
A arquitetura é planejada para funcionar conforme a seguinte especificação\cite{albion_online_unite}:

\begin{itemize}
  \item O mundo é distribuído sobre os vários servidores de mapas. Cada microsserviço gerencia uma região do mundo, denominado \textit{chunk}.
  \item Jogadores mudam a conexão com os microsserviços quando estão posicionados na borda de um \textit{chunk}.
  \item A autorização de acesso aos microsserviços é obtido pelo banco de dados.
  \item O coordenador do mundo é responsável por tudo que seja de escopo global (\textit{e. g.,} Grupos, chat global, guildas, \textit{etc.}).
\end{itemize}

A arquitetura Knowles (Figura~\ref{fig:knowles}) é distribuída em diversos microsserviços, assim como a arquitetura Salz (Figura~\ref{fig:salz}).
%
A diferença em comparação a arquitetura Salz (Figura~\ref{fig:salz}) está na decomposição da arquitetura para outros microsserviços e a conexão direta entre esses microsserviços e o cliente.
%
O principal objetivo dessa arquitetura é facilitar a manutenção e desempenho de reinicialização do macrosserviço~\cite{stephenclarkewillson2017}.
%
Guild Wars 2\footnote[3]{Guild Wars 2: \url{https://www.guildwars2.com}} é um jogo que executa sobre a arquitetura Knowles.
%
Ele é popularmente conhecido por ter seus serviços sempre ativos, visto que a arquitetura possibilita desativar pequenos pedaços do serviço para manutenções básicas e a sua reinicialização é rápida para manutenções críticas.
